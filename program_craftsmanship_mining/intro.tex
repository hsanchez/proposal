\chapter{Introduction}{}
\label{sec:intro}

\lettrine[lraise=0.1, nindent=0em, slope=-.5em]{P}{ROGRAM DESIGN}  plays an 
important role in software development. However, it is both a complex and 
creative process, which can only be performed by developers after much 
practice (over many years). In today’s modern society, while the number of 
citizen programmers is growing at an astounding rate, the percentage of the 
population with a well-rounded program design experience remains in the single 
digits. This means that we are witnessing a growing gap between need and capacity. 
We can meet some of the need by using configurable apps with canned behavior, but 
these restrict creativity and problem solving of their fiddlers. Therefore, it is 
critical that find ``programmatic'' way in which non-traditionally educated 
programmers can borrow the same skills and creativity as the formally trained pros.   


In a new research direction, we term Program Craftsmanship Mining, we aim at 
making some of these abilities, which allow star programmers to succeed, available 
to citizen programmers by applying machine learning to program craftsmanship. We 
will exploit techniques, such as structured prediction, nonparametric bayesian 
analysis, and program induction to facilitate useful interactions for program 
craftsmen. Additionally, it will exploit ready available Web data (e.g., GitHub 
repositories) to foster design inspiration when innovation is required.

Program Craftsmanship Mining enables computers to join developers on multiple 
development activities as equal partners—knowledgeable, creative, and proactively 
responsible. It does it by learning and then operationalizing a type of human-like 
wisdom geared towards program design construction.  Program craftsmanship mining will 
manifest this human-like wisdom throughout development, in creativity and 
proactivity, and within the system’s own structure and behavior. 
