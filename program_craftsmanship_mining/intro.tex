\chapter{Summary}{}
\label{sec:intro}

\section*{Overview} % (fold)
\label{sec:overview}

\startWith{w}{riting code} is not a manufacturing process, it's not 
always craftsmanship, it's design. Design is where developers add 
value faster than they add cost. However, design is both a complex 
and creative process, which can only be performed by developers after 
much practice (over many years). In today’s modern society, while the 
number of people with programming ability is growing at an astounding 
rate, the percentage of the population with a well-rounded program 
design experience remains in the single digits. This means that we are 
witnessing a growing gap between need and capacity. One can meet some of 
the need by using configurable programs with canned behavior, but these 
restrict creativity and problem solving to those anticipated by the 
creators of those programs. Therefore, it is critical that we find 
programmatic ways in which both traditionally and non-traditionally 
educated programmers can express themselves by borrowing the same skills 
and creativity of the formally trained pros.

In a new research direction, we term Program Design Mining, we propose  
making some of the abilities\footnote{e.g., molding programs via idioms, 
coding principles, and design patterns}, which allow star developers to 
succeed, available to citizen programmers by applying machine learning 
to program design. \todo{List these abilities: e.g., idioms, design 
patterns, coding principles} Using ready available Web data (e.g., 
GitHub repositories), we will exploit machine learning techniques, such 
as structured prediction, nonparametric bayesian analysis, and program 
induction to enable new interaction mechanisms for program design.

All of these machine learning techniques leverage structure regularities 
that are intrinsic to program designs. Well-written (object-oriented) 
programs, for example, exhibit many structural regularities\footnote{These 
regularities explicitly molds programs with design and coding principles that 
intend to improve their quality} and their identification and manipulation 
improves with experience. Machine learning can eliminate the need for 
gaining design experience. On the Web data, every Java class is associated 
with an abstract syntax tree. This information can be used along with the 
change history of this class to reason about and manipulate such structural 
regularities, closer to way experienced developers will use. 

With Program Design Mining, we will entirely decouple the ability of 
designing quality programs from years of experience. Design wisdom will 
not come with time. It will come with new design interactions (e.g., code transformations) learned from data. Developers will be able to evolve 
their programs with sound design and coding principles. At every 
opportunity, they will try a different design alternative and experience 
the results. Program design mining will exploit ready available Web data 
to foster design inspiration when innovation is required.

In what follows, we will cover the intellectual merit of data-driven 
approaches for program design, as well as the broader impact of Program 
Design Mining.

\section*{Intellectual Merit} % (fold)
\label{sec:merit}

This project proposes a novel approach for program design and 
evolution. Encouraged by many recent advances in code transplantation, 
data-driven Web design, and software evolution simulation, we 
present Program Design Mining. The key innovations behind this 
approach include:

\begin{enumerate}
	\item Data-driven program design. Establish new design workflows (or code 
	transformations) from existing design knowledge. These workflows allow 
	citizen programmers to automatically transfer the structural regularities 
	from a well-written program into the style and structure of another.  
	\item Operationalize learned regularities and experience the results in 
	real time. On our collected Web data, for example, every Java class is 
	associated with an abstract syntax tree. Along with its change history, 
	we can use this abstract representation to reason about and manipulate 
	programs with \textit{similar} structures.
	\item Mixed-initiative program design evolution (we call design bouncing).     
\end{enumerate}

\section*{Broader Impact} % (fold)
\label{sec:impact}

Computer systems affect most parts of our lives. Countless examples of 
the grave consequences caused by poorly crafted software systems include 
severe financial damages or even the loss of lives. With the looming 
advent of Citizen Programming, the amount of code to be produced by 
opportunistic programmers grows at a staggering pace, demanding for more 
automated tools to ensure software quality. This project contributes a 
data-driven program construction technique that will help people with 
programming ability to write code closer to what an experienced programmer 
would write. Further, the project will create a suite of benchmarks to 
serve as baseline for other automated program construction tools.

SRI International and University of California Santa Cruz make considerable 
contributions to education, research, and technology transfer to industry 
through its freely distributed tools and academic visitor programs such 
as include summer internships for graduate students that encourage applications 
from minority students.