\chapter{Project Description}{}
\label{sec:related}

\startWith{d}{evelopers} frequently rely on templates when designing 
programs. While templates provide a simple mechanism for quick generation of 
commonly used code, their rigidity limits customization and creativity 
to those anticipated by the template creator. Program Design Mining lifts 
this rigidity by allowing the structural regularities of any well-written 
programs to serve as a design template. Program Design works by matching 
statically and semantically similar elements in programs to create coherent 
mappings between them.


programmatic ways in which both traditionally and 
non-traditionally educated programmers can borrow the same skills and 
creativity of the formally trained pros and can still express themselves.  

Program Design Mining will implement a data-driven approach for 
program design. This approach introduces a markup language for 
representing templates and specifying integration instructions 
for code snippets

\begin{enumerate}
	\item A new structured prediction algorithm for rapidly transferring 
	the structural regularities of well-written (object-oriented) programs 
	into the style and structure of another one.
	\item A new program induction algorithm for learning and operationalizing 
	(composition or remix) learned templates to foster program design 
	inspiration when innovation is required.
	\item A new semi-automatic (human-in-the-loop) program evolution algorithm 
	for bouncing program design alternatives (and evolving programs). 
\end{enumerate}

State of the art data-driven tools for program synthesis, program 
repair, and code transplantation, such as X, Y, and Z, focus 
on the code itself and struggle with transferring code from one system 
to another, because of the nontrivial modifications needed to relocate 
unrelated foreign code fragments into new programs. 

SnipMatch introduces a simple markup that allows snippet
authors to specify search patterns and integration instructions.
SnipMatch leverages this information, in conjunction
with current code context, to improve snippet search and
parameterization. 


Instructions for integrating a snippet into the user’s source
code are expressed using a lightweight markup embedded
within the snippet code. The SnipMatch markup is similar
to the markup used in Eclipse Templates [9], with several
additions. Eclipse Templates is a built-in feature in Eclipse
that allows users to create and insert code “templates”
(snippets). Like SnipMatch, it allows snippets to be integrated
into the source code by taking into account local
variables and adding missing import statements. 