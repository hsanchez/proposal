\chapter{Related Work}{}
\label{sec:related}

\lettrine[lraise=0.1, nindent=0em, slope=-.5em]{S} {NIPR IS RELATED} to prior work in three areas: example-centric programming; better code search systems; and systems for transitioning code to alternate contexts and/or APIs.

\fancybreak{\pfbreakdisplay}

\subsection{Finding Suitable Examples}
\label{sec:codesearch}

Many systems have been built to aid programmers with finding trustable and relevant example code. This includes systems that have gone on to include communal input or knowledge collaboration to suggest solutions to problems in the programmers' code. These systems differ from \emph{SnipR} in focus and approach. Instead of enabling programmers to examine---by retargeting in advance---whether a ranked search result is really suitable before trying any code integration, these systems simply return many ranked results that programmers have to manually combine and retarget to discover if an example code is a best fit. 

For example, Brandt’s Blueprint system~\cite{Brandt:2010tp} couples Web search with development environments. Assieme~\cite{Hoffmann:2007wo} combines documentation search results with code snippets of the relevant functions currently in use. Stylos's Mica system~\cite{Stylos:2006gu} integrates search for documentation and example source code. Hartmann's HyperSource system~\cite{Hartmann:2011ii} associates browsing histories with source code edits. Hartmann's HelpMeOut system~\cite{Hartmann:2010hx} aids the debugging of code-related error messages by suggesting solutions that peers have applied in the past. McMillan's Portfolio and Source Code Recommender systems~\cite{McMillan:2011wq, McMillan:2012dj} combines mining written code specifications and open sourced code to recommend source code modules relevant to the application under development.
% dont trust me just because I am telling u to do it. E.g., dont trust this code is relevant just because there is a score that tells u that.

\fancybreak{\pfbreakdisplay}

\subsection{Code Search Engines}
\label{sec:searchengines}
