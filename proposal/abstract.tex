\begin{abstract}
An essential part of knowing what kind of search-driven tools can help developers find suitable code is knowing precisely why search-driven development is so difficult. In a typical code search process, developers deal with a lot of uncertainty. They are given no guidance as to which result may be a best fit for their code, beyond the existing ranking values. As a result, they cannot plan the impact of choosing a code example on their current work until they have tried it. But by then, it’s too late. The disturbance in the developers’ workflow has already occurred and resulted in time wasted.

This proposal sketches a research path that seeks to examine the above issue, based on the observation that when developers include code retargeting in their code search practice, they can confidently justify the suitability of found examples. To support this thesis, this proposal introduces a new approach in search-oriented architecture, called the Snippet Retargeting Approach, or simply \uppercase{SNIPR}. \uppercase{SNIPR} complements code search with code retargeting capabilities. The intent of these capabilities is to help expedite the process of determining if a source code is a best fit---i.e., suitable---by giving developers the means to ``wipe the dirt off a code example so they can see beyond.''

With \uppercase{SNIPR}, developers engage in a virtuous loop where they find code, retarget code, and select only choices they can justify as suitable. That is, the found examples are retargetable and the cognitive distance to understand how to use them is minimal.
\end{abstract}



