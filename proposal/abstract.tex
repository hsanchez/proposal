\begin{abstract}
An essential part of knowing what kind of search-driven tools can help developers find suitable code is knowing precisely why search-driven development is so difficult. In a typical code search process, developers deal with a lot of uncertainty. They are given no guidance as to which result may be a best fit for their code, beyond existing ranking values. As a result, they cannot plan the impact of a code example on their current work until they have tried it. But by then, it’s too late. This limitation is one of the reasons why search-driven development is so cumbersome, and ultimately can be a drain on one of the most precious resources in software development: time.

This proposal sketches a research path that seeks to examine the above issue, based on the observation that when developers include code retargeting in their code search practice, they can confidently justify the suitability of found examples and thus reduce these uncertainties. To support this thesis, this proposal introduces a new approach in search-oriented architecture, called the Snippet Retargeting Approach, or simply \uppercase{SNIPR}. \uppercase{SNIPR} complements code search with code retargeting. It provides a general technique for retargeting structure between related snippets.  Its intent is to help expedite the process of determining if a source code is a best fit---i.e., suitable---without developers leaving the search interface. With \uppercase{SNIPR}, developers engage in a virtuous loop where they find code, retarget code, and select only choices they can justify as suitable. The consequence effect is a faster search process.  

\end{abstract}



