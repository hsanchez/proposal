\begin{abstract}
The recent rise of Internet-scale code search engines---e.g., Ohloh code, Portfolio---has given search-driven development more diverse sources of information---e.g., from open source libraries to online snippets from Stackoverflow---and a more open platform---the Internet. This new condition has enabled developers to build applications opportunistically by iteratively finding, and reusing online source code. This opportunistic way of building applications is not easy. This is because search sources are large, in most cases unsuitable, and quite often unrelated. Consequently, if search-driven development were to be established as best practice, then the time involved in deciding a best search result to reuse must be minimized.


This proposal sketches a research path that seeks to examine the above issue, based on the observation that when developers include code retargeting in their code search practice, they can confidently justify the suitability of found examples and thus reduce their code searching efforts. To support this thesis, this proposal introduces a new approach in search-oriented architecture, called the Snippet Retargeting Approach, or simply \uppercase{SNIPR}. \uppercase{SNIPR} complements code search with code retargeting capabilities. The intent of these capabilities is to help expedite the process of determining if a found example is a best fit---i.e., suitable---by giving developers the means to explore code modification ideas in place, without requiring to leave the search interface. With \uppercase{SNIPR}, developers engage in a virtuous loop where they find code, retarget code, and select only code choices they can justify as suitable. This assures an immediate feedback on the retargeted examples, which saves valuable human time searching for some appropriate code. 

\end{abstract}



