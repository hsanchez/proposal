
\newpage
\setcounter{page}{1}

%\section{References}
\addtocounter{section}{1}
\def\refname{\thesection~~~\bf Data Management Plan \mytitle}
\section*{Data Management Plan \mytitle}
%\def\refname{\bf Data Management Plan \mytitle}

%Plans for data management and sharing of the products of research. Proposals must include a supplementary document of no more than two pages labeled “Data Management Plan”. This supplement should describe how the proposal will conform to NSF policy on the dissemination and sharing of research results (see AAG Chapter VI.D.4), and may include:
\begin{description}
\item[Types of data.]
%the types of data, samples, physical collections, software, curriculum materials, and other materials to be produced in the course of the project;
 The proposed project is expected to produce the following kinds of data:
 \begin{itemize}
  \item Software tools.
  \item Research papers.
  \item Development blog.
  \item Tool documentation and examples.
  \item Slides of presentations.
 \end{itemize}
 All the above artifacts will be made available through a project website maintained at {\url{http://pdm.vesperin.com}}.
 
\item[Standards.]
%the standards to be used for data and metadata format and content (where existing standards are absent or deemed inadequate, this should be documented along with any proposed solutions or remedies);
Software tools, along with their sources, will be made available as tar gzipped files and on the software repository site, GitHub. Research papers, presentation slides, and tool documentation will be published as PDF files. New benchmark programs will be contributed to the public SVCOMP repository and added to the project's own benchmark
repository at \url{https://github.com/jayhorn/benchmarks}. 
 
%
%policies for access and sharing including provisions for appropriate protection of privacy, confidentiality, security, intellectual property, or other rights or requirements;
\item[Policies for access and sharing.]
Project results will be publicly available and source code will be made available. 

%
%policies and provisions for re-use, re-distribution, and the production of derivatives; and
%
\item[Policies for re-use.]
Project results will be freely available for use/re-use by the research community.
%plans for archiving data, samples, and other research products, and for preservation of access to them.

\item[Plans for archiving and sustaining.]
SRI International maintains department, division, and project web page servers. Websites are backed up onsite and offsite. The PI's homepage, {\url{http://csl.sri.com/~schaef/}}, has been actively maintained for several years, and we expect to continue making the site
available after the proposed project ends.

\end{description}
%Data management requirements and plans specific to the Directorate, Office, Division, Program, or other NSF unit, relevant to a proposal are available at: http://www.nsf.gov/bfa/dias/policy/dmp.jsp. If guidance specific to the program is not available, then the requirements established in this section apply.
%
%Simultaneously submitted collaborative proposals and proposals that include subawards are a single unified project and should include only one supplemental combined Data Management Plan, regardless of the number of non-lead collaborative proposals or subawards included. FastLane will not permit submission of a proposal that is missing a Data Management Plan. Proposals for supplementary support to an existing award are not required to include a Data Management Plan.
%
%A valid Data Management Plan may include only the statement that no detailed plan is needed, as long as the statement is accompanied by a clear justification. Proposers who feel that the plan cannot fit within the supplement limit of two pages may use part of the 15-page Project Description for additional data management information. Proposers are advised that the Data Management Plan may not be used to circumvent the 15-page Project Description limitation. The Data Management Plan will be reviewed as an integral part of the proposal, coming under Intellectual Merit or Broader Impacts or both, as appropriate for the scientific community of relevance.
 
