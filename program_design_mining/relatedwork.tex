The literature on verification of data-structure manipulating 
programs is rich. Separation logic~\cite{Reynolds02,IshtiaqO01} has
emerged as a succinct heap logic that is amenable to local
reasoning. Separation logic extends Hoare Logic to support reasoning
about shared mutable data structures. 
Initial works use {\it frame rules} for modular
verification of heap-manipulating programs 
against user-supplied
specifications~\cite{BerdineCO05}.
More recently, {\it
  bi-abduction}~\cite{calcagno2009compositional} 
   has been proposed to automatically
infer pre/post specifications on data structure shapes. Pek
et al.~\cite{PekQM14} describe a technique that adapts ideas of
natural proofs for separation logic~\cite{Qiu0SM13} to automate
verification proofs of data-structure manipulation in programming
language. jStar~\cite{DistefanoP08} is a tool for Java that uses
abstract shape analysis for inferring invariants. Instead of using a
specific purpose-oriented logic, the proposed work is based on a generic
underlying logic to capture the verification conditions (i.e. Horn
clauses). In \cite{McCloskeyRS10} the authors describe a technique based on
abstract interpretation to reason about programs that use data structures
for which a proof of correctness requires a combination of heap and numeric
reasoning. Such technique is not fully automatic, the user needs to provide some initial 
predicates, and it does not scale to large programs since it requires inlining of 
function calls. 
Nguyen et al.~\cite{NguyenDQC07} describe a logic-based approach that involves providing
an entailment procedure. The logic allows for user-defined, well-founded inductive
predicates for expressing shape and size properties of data structures. Their
approach can express invariants that involve other numeric properties of data
structures, such as heights of trees. However, their approach
is limited to separation logic, and the user needs to provide
shape and size predicates of data structure. 


Numerous approaches have been presented that generate quantified
inductive invariants. For example, the work
of Bjorner et al.\ \cite{BjornerMR13}, in which the Horn solver is
guided by constraining the form of the proof, enables it to find
suitable (quantified) invariants even in challenging cases. A related
technique for approximating array behavior was presented
recently~\cite{DBLP:conf/sas/MonniauxG16}.  A number of abstract
interpretation methods have domains that represent universally
quantified facts \cite{GulwaniMT08,Cousot15}.  The proposed work, like
in \cite{BjornerMR13}, avoids the explicit construction of
abstract post operators, and of widening and refinement procedures needed in
these approaches. The work in \cite{LarrazRR13} synthesizes a class of
universally quantified linear invariants of array programs. Our
technique aims to synthesize a general class of invariants.

The work on heap invariants is inspired by techniques based on dependent
refinement and liquid types~\cite{RondonKJ08}.  Refinement-type
  systems are a lightweight form of dependent types, where the
additional precision in the system is limited to constraints, or
predicates, expressed in a logic composed of decidable theories.
Variations of refinement types and their applications have been
studied in many settings
(e.g. \cite{FreemanP91,XiP99,KnowlesF10,BengtsonBFGM08}).  
%
Liquid types~\cite{RondonKJ08} allow automatic inference of dependent
types precise enough to prove a variety of properties.  Our approach
is formulated in terms of invariants (in contrast to types), and
mainly focuses on deriving heap invariants for
object-oriented programs.
  
The use of push and pull in this proposal is similar to  unfold/fold
techniques used for program transformation~\cite{BurstallD77} on purely
functional programs.
A similar technique called unroll/roll was later used in alias
types~\cite{WalkerM00} to manually witness the isomorphism between a recursive
type and its unfolding.

The proposed approach to find invariants for linked data structures is 
inspired by k-index invariants~\cite{Sanchez:2012:IGP:2414936.2414950} which can be seen as an extension of thread-modular model 
checking~\cite{Flanagan:2003:TMC:1767111.1767125}. These approaches focus
on parallel systems. This proposal aims to transfer this technique to handle
linked heap data structures.


