
\setcounter{section}{1}
\section{Summary \mytitle}

% Project Summary: The Project Summary consists of an
% overview, a statement on the intellectual merit of the proposed
% activity, and a statement on the broader impacts of the proposed activity.
%
% Please provide between 2 and 6 sets of keywords at the end of the
% overview in the Project Summary. CISE personnel will use this information
% in implementing the merit review process. The keywords should describe the main scientific/engineering areas explored in the proposal. Keywords should be
% prefaced with "Keywords" followed by a colon and each keyword set should be
% separated by semi-colons. Keywords should be of the type used to describe research
% in a journal submission. They should be included at the end of the overview
% in the project summary and might appear, for example, as Keywords: energy-aware
% computing; formal logic; computer graphics; sensor networks; information visualization;
% privacy. "Breakthrough proposals" should have the word "breakthrough" as the
% first keyword in the submitted list of keywords.

% self contained description of activity that would result if proposal
% were funded objective methods employed

{\bf{Overview:}}
%
Writing code is not a manufacturing process, it's not always craftsmanship, it's
design. Design is where developers add value faster than they add cost. However,
design is both a complex and creative process, which can only be effectively
performed by developers after much training (over many years). As such, while it
would seem a natural fit, there is limited work, from a machine learning
perspective, on reducing the gap between acquiring this design ability and the
time it takes to fine tune it. With this project, we address this shortcoming,
by applying a data-driven approach for program design, called \pdm.

With \pdm, we will examine ways in which we can get software to borrow
the same skills and creativity of the formally trained pros, and how
we can programmatically extend on them to help both traditionally and
non-traditionally educated programmers to creatively design quality programs.
\pdm is grounded on the premise that every well-written program, on the
Web (e.g., GitHub software repositories), represents a concrete example
of human creativity and design wisdom. As such, we can leverage such a
readily available wisdom to establish new design interactions by
identifying, operationalizing, and then manipulating structural
regularities of (object-oriented) programs, such as idioms, coding rules,
and design patterns.

\medskip\noindent{\bf{Keywords}}: program design; design tuning;
design transfer; software repository mining.

\medskip\noindent
{\bf{Intellectual Merit:}}
%
This project proposes a novel approach for program design. Encouraged
by many recent advances in code transplantation, data-driven Web design,
and software evolution simulation, we present \pdm. The key innovations
behind this approach include:

\begin{enumerate}
	\item Data-driven program design for (object-oriented) Java programs.
	\item Implementation of semi-automatic program design tuning and evolution.
	\item Learning and operationalization of structural regularities of programs.
\end{enumerate}

\noindent
The project will integrate the developed \pdm technique into an existing
integrated development environment and then evaluate its effectiveness
on a series of benchmarks. The goal is to advance the effectiveness and
performance of Java program design tools and provide a new set of benchmarks
for a Java program design competition (similar to the software design
competitions held by TopCoder and Microsoft).


\medskip\noindent
{\bf{Broader Impact:}}
Software systems exist in our lives in hidden, popularized, and ubiquitous
ways. Failures in these systems can be costly in terms of money, time, and
human life. With the rise of Opportunistic Programming, the amount of code written
by opportunistic programmers is growing at an astounding rate, demanding for
more automated tools to ensure software quality. This project contributes a
data-driven program design technique that will help opportunistic programmers
to write code closer to what an experienced programmer would write.

SRI International make considerable contributions to education, research,
and technology transfer to industry through its freely distributed tools
and academic visitor programs such as include summer internships for graduate
students that encourage applications from minority students.
