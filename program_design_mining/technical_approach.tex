\subsection{Technical Approach}
\label{sec-extension}

\newcommand{\true}{\ensuremath{\mathsf{true}}\xspace}
\newcommand{\false}{\ensuremath{\mathsf{false}}\xspace}

\newcommand{\smallpar}[1]{\medskip\noindent\emph{#1}:}

\newcommand{\expr}{\ensuremath{\textit{t}}\xspace}
%local var
\newcommand{\lvar}{\ensuremath{\textit{x}}\xspace}
\newcommand{\lvarp}{\ensuremath{\textit{y}}\xspace}
%pointer var
\newcommand{\pvar}{\ensuremath{\textit{p}}\xspace}
\newcommand{\qvar}{\ensuremath{\textit{q}}\xspace}
%class var
\newcommand{\cvar}{\ensuremath{\textit{c}}\xspace}

\newcommand{\fvar}{\ensuremath{\textit{f}}\xspace}

\newtheorem{theorem}{Theorem}[section]
\newtheorem{lemma}[theorem]{Lemma}

\lstset{language=Java,
basicstyle=\ttfamily\small,
frame=none,
rulecolor=\color{gray},
breaklines=true,
tabsize=2,
morekeywords={assume, assert},
captionpos=b,
numbers=none,                   
keywordstyle=\color{blue},
commentstyle=\color{green2} }



This project will develop a new encoding of Java-like languages into Horn clauses 
using the concept of heap invariants. The encoding will be implemented as part
of our \jayhorn tool~\cite{jayhorn16}. \jayhorn is a software model 
checking tool for Java. It takes single-threaded and reflection-free Java bytecode
as input and tries to verify that a user-provided \textsf{main} method cannot throw an
undeclared exception. To that end, \jayhorn performs the following steps, which have been implemented as part of our preliminary work:
\begin{enumerate}
\item Elimination of implicit control-flow in Java bytecode.
\item Simplification of heap access by introducing pull
and push statements.
\item Abstraction of heap using heap invariants.
\end{enumerate}
We briefly discuss this preliminary work to introduce the relevant terminology
before outlining the planned research. A more detailed description of these 
steps can be found in~\cite{jayhorn16} and in the \jayhorn development 
blog\footnote{\url{jayhorn.github.io/jayhorn/blog}}. 


\paragraph{Elimination of implicit control-flow.}
\jayhorn uses the Soot bytecode library~\cite{vall99soot} to obtain a 
Jimple representation of the analyzed program, and performs a series of program 
transformations to simplify the program and add runtime checks for several implicit
assertions, such as Null\-Pointer\-Exception or 
Array\-Index\-Out\-Of\-Bounds\-Exception before translating the 
simplified program into a system of constrained Horn clauses.
\begin{figure}[tb]
\begin{center}
\begin{align*}
\textit{Program} \; ::= & \; \textit{Method}^* \\
\textit{Method} \; ::= & \; f(\lvar_1, \ldots,
\lvar_n)\{\textit{Stmt}\textbf{;}^*\}\\
 \textit{Stmt} \; ::= & \; \textit{label}:\; \textit{Stmt} 
\; |   \; \textbf{goto} \,  \textit{label}^+
\; |  \; \lvar:= \textit{t}
\; |   \; \textbf{load}(\lvar, \pvar.\fvar) 
\; |   \; \textbf{store}(\pvar.\fvar, \expr) \\
| & 
 \; (\lvarp_1, \ldots, \lvarp_m) := f(\expr_1, \ldots, \expr_n)
 \; | \; \textbf{return} \; (\expr_1, \ldots, \expr_m)
\\
| & 
 \; \textbf{assume}(\expr) \;  | \; \textbf{assert}(\expr) 
 \;  | \; \textbf{havoc}(\lvar_1,\ldots,\lvar_n) 
\end{align*}
\end{center}
\caption{Syntax of the simplified bytecode used by \jayhorn. To model abstraction, e.g., from library calls whose code we cannot access, the syntax also provides a \texttt{havoc} statement to represent non-deterministic updates.\label{fig-syntax}}
\end{figure}

In the transformed program, implicit control-flow stemming from the dispatch 
of virtual methods and exceptional flow is replaced by explicit 
control-flow to simplify processing.
The resulting language is a simplified version of Java bytecode that can be described 
using the syntax in Figure~\ref{fig-syntax} where we use the following 
conventions: $\lvar, y, z$ refer to local variables; $\pvar, r$ are 
local variables that refer to
a memory location on the heap, and $\expr, s$ are side-effect-free
expressions over variables and constants.  A program in this language
consists of a set of methods (or functions) with one distinct
main-method. Each method consists of a sequence of (possibly
labeled) statements. Labels are symbolic representations of the
program counter associated with a particular statement. The first
statement in the sequence is the entry point of the method. 

Methods
return a pair consisting of the last thrown exception 
(or \texttt{null} if no exception has been thrown) and the actual
return value (or \texttt{null} if the method does not return a
value). Encoding exceptional flow through additional return values
is a known technique used in tools like ESC/Java or 
Spec\#. This encoding
is sound as long as all statements that may throw an exception
explicitly update this return value and leave the method, and if 
the value is checked for nullness after every method call.
In this encoding, 
the last statement on every path through a method is a 
return-statement. Control
flows from one statement to the next as the program counter
increases. 
The statement $\textbf{assert}(\expr)$ terminates
an execution erroneously if $\expr$ evaluates to $\mathit{false}$
and has no effect otherwise. 
The statement $\textbf{assume}(\expr)$ blocks the execution
if $\expr$ evaluates to $\mathit{false}$ on the current state and has
no effect, otherwise. Like in other verification languages, such as
 Boogie~\cite{Leino:2010:PIV:2175554.2175588}, \jayhorn uses
\textbf{assume} and non-deterministic goto to model
conditional choices and loops. Assignment statements $\lvar:=\expr$
update a variable~$\lvar$ to the value of $\expr$.  The statement
$\textbf{havoc}(\lvar_1,\ldots,\lvar_n)$ assigns non-deterministic
values to the variables $\lvar_1,\ldots,\lvar_n$. 


Initially, heap interaction is modeled, like in Java bytecode, using
\textbf{load} and \textbf{store}.
The statement $\textbf{load}(\lvar, \pvar.\fvar)$ loads the value of
field \fvar of \pvar into a local variable
\lvar; accordingly, $\textbf{store}(\pvar.\fvar, \expr)$ updates field
\fvar of object~\pvar to the value of expression \expr. For simplicity, 
assume, for static fields, a global $\lvar$ that can be used as base.

A method
call $(\lvarp_1, \ldots, \lvarp_m) := f(\expr_1, \ldots, \expr_n)$ to
a method $f(\lvar_1, \ldots, \lvar_n)$ executes the body of the method
$f$ until a $\textbf{return}\;(\textit{ret}_1, \ldots, \textit{ret}_m)$ 
statement is
reached, then control returns and the values of the
expressions~$\textit{ret}_1, \ldots, \textit{ret}_m$ are assigned to $\lvarp_1,
\ldots, \lvarp_m$. As discussed earlier, one of the return expressions is
used to store the last thrown exception. Call statements will be followed
by conditional choices to check if the call terminated exceptionally before
evaluating its other returns values.

Virtual method calls are resolved by explicitly enumerating
all possible calls and guarding them with the appropriate \texttt{instanceof}
test. Note that this approach is sound but inefficient. Later we discuss
how \jayhorn is extended with a points-to analysis that can be used to
prune irrelevant cases. 

This language does not provide primitives 
for object creation; instead, allocation is encoded by passing around a counter
that indexes the next free memory block, while constructors are turned
into ordinary methods.


Semantically, the state of a program is a triple $(S, H, pc)$, where
$S : \textit{Vars} \rightarrow \mathbb{Z}$ is the local store, $H :
\mathbb{Z}\times \textit{fields} \rightarrow \mathbb{Z}$ is the heap;
and $pc$ is the program counter pointing to the current
statement. Execution of statements updates the program state in the
expected way; in particular, an assignment statement $\lvar:=\expr$ in
a state $(S, H, pc)$ updates the local store so that $(S[\lvar
\mapsto \expr^S], H, pc+1)$ where $\expr^S$ denotes the
  value of expression $\expr$ under store~$S$; a statement
$\textbf{load}(\lvar, \pvar.\fvar)$ has the effect $(S[\lvar \mapsto
H(S(\pvar).\fvar)], H, pc+1)$; a statement
$\textbf{store}(\pvar.\fvar, \expr)$ has effect $(S,
H[S(\pvar).\fvar\mapsto \expr^S], pc+1)$. For a heap entry
  $H[\pvar.\fvar]$ with $\pvar \in \mathbb{Z}$ and
  $\fvar\in\textit{fields}$, we assume that $\pvar$ is associated with
  a reference type $\cvar$ that contains the field $\fvar$.  Since we
  generate programs from type-checked code, this will be the case by
  construction. A method call $(\lvarp_1, \ldots, \lvarp_m) :=
f(\expr_1, \ldots, \expr_n)$ will 1) store the program counter~$pc$
and values of local variables in fresh temporary variables, 2)
assign the values of the arguments $\expr_i$ to the corresponding
formal arguments $\lvar_i$, and 3) jump to the invoked method.
After return, the program counter and all local variables are
restored.

%
A path in a program is a sequence of statements starting from the first
statement in the \textsf{main} method of a program. A path is complete if it
reaches the end of the \textsf{main} method. A trace is an initial state and the
path of all statements that are executed when starting the program from this
state. A trace terminates if its path is finite. 
A trace is infeasible if its path ends on an assumption violation. 
A program is safe if it has no feasible error trace.


% We model exceptional returns of a method by adding an additional return value
% (our language allows tuples as return values). This return value is either 
% the \texttt{null} reference or the reference of the last thrown exception. 
% On the caller site, we then have to distinguish if this value is non-null and
% forward the exception accordingly. With this modeling of exceptional flow, 
% an \emph{error trace} is a trace where the exceptional return value is non-null and 
% the type of the return value is not in the set of types declared in the 
% throws-clause of \textsf{main}\footnote{Arbitrary safety properties can 
% be expressed using assertions since Java assertions are turned into 
% exceptions by the compiler.}.

The presented language simplifies the verification process by making exceptional
flow and dynamic dispatch of virtual invocations explicit. Further, it makes
invocations of static initializers and default variable initialization explicit. A program in this language that 
does not use the heap can be verified by any off-the-shelf model checker. What makes
the problem complicated is the heap; proving safety requires finding invariants 
for a potentially unbounded number of objects. 

\paragraph{Heap interaction.} To make the language more accessible to
model checking, \jayhorn bundles field accesses by pulling 
all fields of an object into the local state in one transaction before
modifying its fields, and pushing all fields back in a single 
transaction after it is done with our modifications. To that end, two
adding two additional statements, \textbf{pull} and \textbf{push} are 
added to language:
\begin{equation*}
 \textit{Stmt}_{\mathit{PP}} \; ::= \; \textit{Stmt}
 \; | \; \textbf{push}_\textit{\cvar}(\pvar,
     \expr_1,\ldots,\expr_n)
 \; | \; \textbf{pull}_\textit{\cvar}(\lvar_1,\ldots,\lvar_n,
\pvar)
\end{equation*}
Here, \cvar is a reference type with fields $\fvar_1,\ldots,\fvar_n$,
and \pvar is a reference of type \cvar.  The intuition of the new
statements is that, the number of heap interactions can be reduced
by bundling them. Instead of having to handle separate heap accesses
for each field using \textbf{load} and \textbf{store}, 
the state of an object is pulled into local variables, and
all modifications to that object can be performed locally until
the changes need to be made visible using a push.


The naming similarity to push and pull
in \textsf{Git} is intentional: pull fetches the current version of a
reference type from the heap and push communicates the local changes
back to the heap. This idea is not to be confused with the concept of
pack and unpack in~\cite{leino2004object}. The transformation from
\textbf{load}/\textbf{store} to \textbf{push}/\textbf{pull} will
prepare the ground for the later step of describing object states
using heap invariants.

For program counter~$pc$ pointing to a statement
$\textbf{push}_\cvar(\pvar, \expr_1,\ldots,\expr_n)$, where $\pvar$ is
a variable of reference type~\cvar\ with fields
$\fvar_1,\ldots,\fvar_n$, and $\expr_1,\ldots,\expr_n$ are expressions
whose type conforms to $\fvar_1,\ldots,\fvar_n$, the semantics of
$\textbf{push}$ is to simultaneously update the fields
$\fvar_1,\ldots,\fvar_n$ of $p$ to the values of
$\expr_1,\ldots,\expr_n$:
\[
(S,H,pc) \;\rightarrow\;
(S, H[S(\pvar).\fvar_1 \mapsto \expr_1^S,\,
S(\pvar).\fvar_2 \mapsto \expr_2^S,\, \ldots
\ldots], pc+1)
\]
%
Similarly, the semantics of
$\textbf{pull}_\cvar(\lvar_1,\ldots,\lvar_n, \pvar)$ is to
simultaneously load the contents of the fields
$\fvar_1,\ldots,\fvar_n$ of $p$ and store them in the local variables
$\lvar_1,\ldots,\lvar_n$:
\[
(S,H,pc)
\;\rightarrow\;
(S[\lvar_1 \mapsto H(S(\pvar).\fvar_1),\,
\lvar_2\mapsto H(S(\pvar).\fvar_2),\,
\ldots], H, pc+1)
\]

Given this semantics, it is clear that \textbf{load} and
\textbf{store} instructions can be uniformly replaced with
\textbf{pull} and \textbf{push} statements.
% In addition to that, we define a set
%of simplification rules to minimize the number of usages of
%\textbf{pull} and \textbf{push}.
%
Assume \cvar\ is a reference type with fields
$\fvar_1,\ldots,\fvar_n$, and $k \in \{1, \ldots, n\}$ is one of
the field indexes, occurrences of \textbf{load} and \textbf{store} can
then be eliminated using the following replacement rules (in which
$\lvarp_1 \ldots \lvarp_n$ are fresh local variables):
\begin{align}
  \label{eq:loadRepl}
  \textbf{load}(\lvar, \pvar.\fvar_k) &~~\leadsto~~
  \textbf{pull}_\cvar(\lvarp_1,\ldots,\lvarp_n, \pvar);\;\; x:=\lvarp_k
  \\
  \label{eq:storeRepl}
  \textbf{store}(\pvar.\fvar_k, \expr) &~~\leadsto~~
  \textbf{pull}_\cvar(\lvarp_1,\ldots,\lvarp_n, \pvar);\;\;
  \lvarp_k := \expr;\;\;
  \textbf{push}_\cvar(\pvar, \lvarp_1,\ldots,\lvarp_n)
\end{align}



\paragraph{Heap Invariants.}
The representation of heap accesses using $\textbf{push}$ and
$\textbf{pull}$ enables \jayhorn to make the step from precise
program execution semantics to summarization of states using
invariants. 

The summarization is performed by replacing 
$\textbf{push}_\cvar$ and $\textbf{pull}_\cvar$ statements symbolic
invariants representing possible states of objects of
type~$\cvar$. Assume a type $\cvar$ with fields $\fvar_1, \ldots,
\fvar_n$, a \emph{$\cvar$-invariant} is a formula~$\phi_\cvar$ over
pairwise distinct (fresh) variables $\pvar, \lvar_1, \ldots, \lvar_n$,
and will be used to capture the possible states of an object of
type~$\cvar$ at address~$\pvar$. As a convention, for expressions~$s,
\expr_1, \ldots, \expr_n$, we write $\phi_\cvar(s, \expr_1, \ldots,
\expr_n)$ for the result of the substitution~$\phi_\cvar[\pvar/s,
\lvar_1/\expr_1, \ldots, \lvar_n/\expr_n]$.

The elimination of $\textbf{push}_\cvar$ and $\textbf{pull}_\cvar$ 
statements through
the $\cvar$-invariant~$\phi_\cvar$ is performed by an exhaustive
application of the following replacement rules:
\begin{align}
  \label{eq:pushRepl}
  \textbf{push}_\cvar(\pvar, \expr_1,\ldots,\expr_n) &~~\leadsto~~
  \textbf{assert}(\phi(\pvar, \expr_1 \ldots \expr_n))
  \\
  \label{eq:pullRepl}
  \textbf{pull}_\cvar(\lvar_1,\ldots,\lvar_n, \pvar) &~~\leadsto~~
  \textbf{havoc}(\lvar_1,\ldots,\lvar_n);\;\;
  \textbf{assume}(\phi(\pvar, \lvar_1 \ldots \lvar_n))
\end{align}
In other words, instead of actually writing data to memory, the
resulting program asserts that the data satisfies the stipulated
invariant~$\phi_\cvar$; reading data from memory is translated to
generating arbitrary values satisfying the invariant. 

% This
% transformation is \emph{sound:}
% \begin{lemma}
%   Let $P_{\mathit{PP}}$ be a program that does not
%   contain~\textbf{load} or \textbf{store} instructions, and
%   $P_{\text{inv}}$ the result of exhaustively applying
%   \eqref{eq:pushRepl} and \eqref{eq:pullRepl}. If $P_{\mathit{PP}}$
%   does not access uninitialized memory, then $P_{\mathit{PP}}$ is
%   correct if $P_{\text{inv}}$ is correct (but in general not vice
%   versa).
% \end{lemma}

\paragraph{Horn encoding.}
At this point, our language contains only local variables and the 
translation into logic becomes simple. To prove the safety of all 
assertions in a program, we encode
the program into a set of constrained Horn clauses (CHC).

A CHC is a formula~$C \wedge B_1 \wedge
\cdots \wedge B_n \to H$, where $C$ is a constraint over variables
and interpreted predicates (e.g.,  $>$ or $+$). Each $B_i$ is an
application of a relational symbol $p(\expr_1, \ldots, \expr_n)$ to
terms $\expr_i$ in first-order logic. We refer to $C \wedge B_1 \wedge
\cdots \wedge B_n$ as the body of a Horn clause. The
head of a Horn clause $H$ is, similar to $B_i$, an application
$p(\expr_1, \ldots, \expr_n)$, or $\mathit{false}$.
A set of CHCs is solvable if there is an instantiation
of the used relational symbols for which all Horn clauses are valid.
As shown in~\cite{verificationAsSMT}, the assertion checking problem
can be reduced to the solvability of Horn clauses.

Given a method $f(\lvar_1, \ldots, \lvar_n)$ with $n$ arguments and
$k$ results. To represent the contract of $f$, we
follow the standard encoding (e.g., \cite{GrebenshchikovLPR12})
and assume two predicates $\mathit{pre}_f$ and $\mathit{post}_f$,
where $\mathit{pre}_f$ has arity~$n$ and $\mathit{post}_f$ has
arity~$n+k$; those predicates will represent the precondition and
postcondition of $f$.
We further assume one predicate $\mathit{pc}_i$ per program location
$\mathit{pc}=i$. The arity $m = n + l$ of this symbol is determined by
the number~$n$ of arguments of $f$, and the number~$l$ of program
variables that can be alive at this program point (live variables can
be computed statically). In addition to that, we create one predicate
$\phi_\cvar$ for each $\cvar$-invariant for each $\cvar$ used in the
program. The arity~$m+1$ of $\phi_\cvar$ is determined by the
number~$m$ of fields in $c$ (one additional argument is needed for the
object reference). These predicates are shared between all methods.

For the method entry point, we introduce one Horn clause connecting
the precondition with the method entry~$\mathit{pc} = 1$:
\begin{align*}
  \mathit{pre}_f(\lvar_1, \ldots, \lvar_n) & ~\to~
  \mathit{pc}_1(\lvar_1,\ldots, \lvar_n, \ldots)% \\
%  \mathit{pc}_n(\lvar_1,\ldots, \lvar_m) & \to \mathit{post}_f(\lvar_1,\ldots,
%  \lvar_m)
\end{align*}

For each statement $s$ at program counter $\mathit{pc}=i$ whose
execution transitions to $\mathit{pc}=j$, a Horn
clauses of the following shape is generated:
%
\[
\mathit{pc}_i(\lvar_1,\ldots, \lvar_n, \ldots) \wedge \psi ~\to~
\mathit{pc}_j(\lvar_1,\ldots,\lvar_n, \ldots) \]
%
The method arguments are always carried through in this clause (since
postconditions can refer to the method arguments), while further
predicate arguments, and the guard $\psi$, depend on the nature of the
statement~$s$. Intuitively, the Horn clause expresses that, if we are
at program counter $i$ and the invariant $\mathit{pc}_i$ and the
condition $\psi$ hold, then we can transition to program counter $j$
where the invariant $\mathit{pc}_j$ has to hold. Additional clauses
are introduced for method calls, \textbf{return}, and \textbf{assert}
statements.

%\medskip
%
The translation from the individual statements to clauses is mostly
straightforward, we only show the most interesting cases.


A statement~$\textbf{assert}(\phi)$ at $\mathit{pc}=i$ is translated
to two Horn clauses:
\begin{align}
  \label{eq:assertFalse}
  \mathit{pc}_i(\lvar_1,\ldots, \lvar_n, \lvarp_1, \ldots, \lvarp_l)
  \wedge \neg \phi &~\to~
  \mathit{false}
  \\
  \notag
  \mathit{pc}_i(\lvar_1,\ldots, \lvar_n, \lvarp_1, \ldots, \lvarp_l) &~\to~
  \mathit{pc}_{i+1}(\lvar_1,\ldots,\lvar_n, \ldots)
\end{align}
where we assume that $\phi$ is formulated over the live
variables~$\lvarp_1, \ldots, \lvarp_l$.

As a special case, for a statement~$\textbf{assert}(\phi(\pvar,
\expr_1 \ldots \expr_n))$ introduced by \eqref{eq:pushRepl},
clause~\eqref{eq:assertFalse} is turned into an implication
\begin{equation*}
  \mathit{pc}_i(\lvar_1,\ldots, \lvar_n, \lvarp_1, \ldots, \lvarp_l) ~\to~
  \phi_\cvar(p, \expr_1 \ldots \expr_n)
\end{equation*}


A call $(\lvarp_1, \ldots, \lvarp_{k'}) := g(\expr_1, \ldots,
\expr_{n'})$ to method~$g$ is represented by two clauses,
one asserting the precondition of $g$, and one applying the
postcondition:
\begin{align*}
\mathit{pc}_i(\lvar_1, \ldots, \lvar_n, \lvarp_1, \ldots, \lvarp_l) &
~\to~ \mathit{pre}_g(\expr_1,\ldots,\expr_{n'}) \\
\begin{array}{@{}l@{}}
  \mathit{pc}_i(\lvar_1, \ldots, \lvar_n, \lvarp_1, \ldots, \lvarp_l)
  \wedge\mbox{}
  \\
  \mathit{post}_g(\expr_1,\ldots,\expr_{n'}, r_1, \ldots, r_{k'})
\end{array}
 &
~ \to~ \mathit{pc}_{i+1}(\lvar_1,\ldots,\lvar_k,
\ldots, r_1,\ldots,r_{k'},\ldots)
\end{align*}
Expressions~$\expr_1,\ldots,\expr_{n'}$ are again formulated over live
variables~$\lvarp_1, \ldots, \lvarp_l$, while the result
variables~$r_1,\ldots,r_{k'}$ in the second clause are used to update
relevant live variables in the post-state.

A statement $\textbf{return} \ (t_1, \ldots, t_k)$ at
$\mathit{pc}=i$ is dually represented as a clause asserting the
postcondition:
\begin{equation*}
\mathit{pc}_i(\lvar_1, \ldots, \lvar_n, \lvarp_1, \ldots, \lvarp_l)
~\to~ \mathit{post}_f(\lvar_1, \ldots, \lvar_n,\expr_1,\ldots,\expr_{k}) 
\end{equation*}

To verify that no bad state is reachable from the program
entry, \jayhorn adds an additional Horn clause: $true \to
\mathit{pre}_{\text{main}}$ to assert that the precondition of
\texttt{main} has to be \textit{true} (i.e., no input can violate an
assertion).
%
% In practice, we have to add a prelude to the Horn clauses representing
% \texttt{main}. The representation discussed above assumes that the
% execution of the program starts with a completely empty heap but, for
% example, the \texttt{main} method of a Java program takes an array of
% strings as input that exists prior to the execution of
% \texttt{main}. Hence, we need to add a short prelude that allocates
% this array and fills it with non-deterministic values.

All these steps are implemented as part of the \jayhorn tool and are
sufficient to verify a set of interesting Java programs. Several limitations to this approach will be addressed by
the proposed research to handle classes of problems that are currently
out of scope.



\subsubsection{Efficient Horn-encoding of Heap Interaction for Java-like Languages}
Using heap invariants as described above is sound, but may become imprecise 
when the analysis too often abstracts the state of an object by its invariant.
Consider the example in Figure~\ref{fig-update}. The class \texttt{A} has a 
constructor where \texttt{x} is first set to zero (implicitly) and then
set to one by the constructor. 


\begin{figure}[ht]
\centering
\begin{minipage}{.3\textwidth}
\begin{lstlisting}
public class A {
 int x;
 public A() {
 	x=1;
 	assert(x==1);
 }
}
\end{lstlisting}
\end{minipage}
\begin{minipage}{.3\textwidth}
\begin{align*}
\textbf{pull}(l_x, this)\\
l_x = 0 \\
\textbf{push}(this, l_x)\\
\textbf{pull}(l_x, this)\\
l_x = 1 \\
\textbf{push}(this, l_x)\\
\textbf{pull}(l_x, this)\\
assert (l_x==1)
\end{align*}
\end{minipage}
\begin{minipage}{.3\textwidth}
\begin{align*}
\textbf{assume} \phi_A(this, l_x)\\
l_x = 0 \\
\textbf{assert}\phi_A(this, l_x)\\
\textbf{assume}\phi_A(this, l_x)\\
l_x = 1 \\
\textbf{assert}\phi_A(this, l_x)\\
\textbf{assume}\phi_A(this, l_x)\\
assert (l_x==1)
\end{align*}
\end{minipage}
\caption{The left column shows code where the constructor of \texttt{A} first performs an implicit update that sets \texttt{x} to zero and then an explicit update that sets \texttt{x} to one. The middle column shows the encoding of
this constructor using pull and push, and the right shows the program after replacing pull and push by heap invariants.}
\label{fig-update}
\end{figure}

On the right, the figure shows the encoding using where pull and push statements have been 
replaced by the heap invariant for class \texttt{A}. Most Horn clause solvers will
instantiate $\phi_A$ by the disjunction $\phi_A(a_0, a_1) := (a_1=0\vee a_1=1)$. This invariant describes all states \texttt{this} can take, but it is not sufficient to verify the assertion at the end of the constructor. 

To address the problem highlighted in this example, this project proposes the following two research tasks:
\begin{enumerate}
\item Minimize the pull and push statements to reduce the points where we abstract from concrete states.
\item Keep a partial order over push statements to allow the Horn solver to infer disjunctives that form an invariant between pairs of push statements. 
\end{enumerate}
The intuition for the first point is that a program does not have to push an 
object until it pulls another object that might interfere with this object, and 
that it does not have to pull  objects that were pulled earlier unless 
there was a push to a comparable type. 

This project will identify cases where pull and push statements can be omitted by using a 
may-alias analysis. This includes defining a set of transformation rules that 
minimize the number of pull and push statements in a program with respect to 
a given entry point, and verifying that these transformations are sound and 
optimal. In particular, the project will work on defining the optimal
number of pull and push statements for a given program under some soundness
assumption.


Instead of using an off-the-shelf alias analysis, this project will investigate
the extent to which Datalog-based alias analysis algorithms, such 
as~\cite{DBLP:conf/aaai/MangalZKNN16,Zhang:2016:QMS:2837614.2837658}, can be
interwoven with the Horn clause encoding used during the verification. Combining
the SAT-based alias checks with the SMT-based path feasibility checks of the
verifier can have a mutual benefit to both algorithms. The alias analysis
gets path-sensitive information from the verifier, and the verifier can quickly
prune certain paths. In previous work~\cite{DBLP:conf/nfm/Schwartz-Narbonne15}, 
we have combined SAT-based reachability checks with SMT-based unreachability 
checks and demonstrated that the combination significantly improves 
performance for complex benchmarks. This project will investigate
if the same effect can be demonstrated for alias analysis.


The second research direction will establish a partial ordering over push 
statements. The goal is to pass the information from the partial ordering
as an additional argument to the push statements, and thus into the predicate
for the heap invariant. A Horn clause solver will then be able to use
this information to derive a disjunctive invariant where each disjunct is
a local invariant that holds between two push statements.



\subsubsection{Implementation of Context-aware Heap Invariants}
Minimizing the number of pulls and pushes makes reasoning about programs 
simpler, since it reduces the points where the analysis can lose precision.
Another way to improve the analysis is to optimize the 
arguments passed to the 
pull and push statements. The Horn solver can use only these arguments
 as vocabulary when searching for suitable heap 
invariants. If the pull and push statements contain too much information, 
they may misguide or slow down the invariant search by introducing too many
disjuncts. If pull and push statements contain too little
information, they restrict the class of invariants the Horn solver can infer. 

This project will explore how the context of a heap access can be used to 
either omit heap accesses or expand the arguments passed into the 
pull and push statements with meta information that can be used to identify
new invariants.

\paragraph{Rich-references.} Some fields of an object do not change 
throughout the lifetime of the object. These fields can be moved into the
local state to ensure that the Horn solver does not lose information about
their value when inferring heap invariants.

This project will investigate the encoding of references
as tuples instead of simple variables to carry values of immutable fields as 
part of the reference. 
Such a tuples can contain the dynamic type of a reference, the size of an array,
and any other value of immutable fields. The tuple encoding allows the analysis
to access the fields in the tuple without performing an additional pull. This
ensures that 
the analysis does not lose information about constant values.
The following 
assignment shows how references can be encoded using tuples:
\begin{align*}
\texttt{ s = ''foo''}\, & ~~\leadsto~~ \, 
\langle id, type, length \rangle = \langle 42, String.class, 3\rangle
\end{align*}
The reference to \texttt{s} is encoded as a tuple that contains the 
immutable information about \texttt{s}. For any Java object, we have the 
immutable information of the location on the heap ($id$) and the
type assigned to it during allocation ($type$). The $id$ element 
will be discussed in more detail later.
Since Strings in 
Java are immutable, their $length$ can also be encoded into the tuple. 
The actual data of the string is also immutable, but since it is
an array of unknown size, it may not be efficient to encode each 
character as part of the tuple. That is, deciding which information to
move into rich references requires tunable metrics so the local
state does not grow beyond what a decision procedure can handle efficiently.

\begin{figure}[ht]
\centering
\begin{minipage}{.3\textwidth}
\begin{lstlisting}
 void example() {
  Strign s = "Hello,";
  Strign t = "world";
  assert s.length!=t.length;
 }
\end{lstlisting}
\end{minipage}
\begin{minipage}{.3\textwidth}
\begin{align*}
s = new() \\
t = new() \\
\textbf{push}(s, String.class, 6, \ldots)\\
\textbf{push}(t, String.class, 5, \ldots)\\
\textbf{pull}(type_s, length_s, \ldots, s)\\
\textbf{pull}(type_t, length_t, \ldots, t)\\
assert (length_s\neq length_t)
\end{align*}
\end{minipage}
\begin{minipage}{.3\textwidth}
\begin{align*}
ref_s = \langle new(), String.class, 6\rangle \\
ref_t = \langle new(), String.class, 5\rangle \\
\textbf{push}(ref_s, \ldots)\\
\textbf{push}(ref_t, \ldots)\\
assert (6 \neq 5)
\end{align*}
\end{minipage}
\caption{The left shows a Java listing that compare two Strings by length. 
The middle shows an encoding where all field accesses are handled
using pull and push. For brevity, we assume that \texttt{new()} is a 
function that returns a fresh memory address. The right shows an encoding
using rich references. Note that, with rich references, the length of the
Strings remains part of the local state and hence the assertion becomes
a comparison of constants.
}
\label{fig-tuple-example}
\end{figure}

The example in Figure~\ref{fig-tuple-example} compares the 
previously described encoding to the rich reference encoding. For brevity,
we assume the existence of a helper function \texttt{new} that returns
fresh heap locations. In practice, handling heap allocation is more complicated. 
The previous encoding, showed in the middle column, updates and
reads fields using pull and push, which may lead to a loss of information, e.g., 
if the Horn solver infers an invariant like $length=5 \vee length=6$. Such an
invariant is valid, but insufficient to verify the assertion. With the rich 
reference encoding in the right column, 
the length of the Strings becomes a local variable as part of the reference.
Hence, the verification procedure does not lose information about the length
of the individual Strings and can verify the assertion. For this example, minimizing pull and push statements could solve the 
problem as well, but it is easy to make the control-flow of the example more
complicated so minimizing the pull and push statements is not sufficient.


This project will investigate different encodings of rich-references
and evaluate their ability to verify interesting Java programs. 
More specifically the project identifies the following two work areas:
\begin{enumerate}
\item Move all immutable fields of an object into rich references. 
\item Re-use the may-alias analysis from the previous step to identify fields that
are constant for a particular reference and move those into the reference as well.
\end{enumerate}
The first step is conceptually straight forward but requires some work to ensure
soundness for the object initialization mechanism of Java. By keeping immutable 
fields as part of the reference instead of obtaining them using pull, the analysis
can preserve their precise value. The second step is conceptually more challenging. Such an analysis has to ensure
that the decision that a field is mutable is in line with other soundness 
assumptions about library calls.


Rich references help the verification procedure to preserve concrete values of
fields that do not change during execution. Another way to improve precision is
to provide additional information to the Horn solver in the pull and push 
statements to find better disjunctive invariants.

\paragraph{Augment Pull and Push Statements.}
The project will investigate different sources of information that can
be encoded into pull and push statements, such as the allocation site 
of an object, the current 
thread, or a counter of allocated objects. The different encodings will be 
benchmarked on a specifically developed set of benchmarks. The benchmarks will
be shared with the community for comparison. 

\begin{figure}[ht]
\centering
\begin{minipage}{.3\textwidth}
\begin{lstlisting}
class Example {
  public int v;

  void set(int val){
    this.v = val;
  }
  void static test() {
    Example a = new Example();
    Example b = new Example();
    a.set(1);
    b.set(2);
    assert a.v!=b.v;
  }
}
\end{lstlisting}
\end{minipage}
\begin{minipage}{.3\textwidth}
\begin{align*}
set( & this, val)  := \\ 
& \textbf{pull}(l_v, this)\\
& \textbf{push}(l_v, val)\\
test(&)  := \\ 
& \textbf{push}(a, 0)\\
& \textbf{push}(b, 0)\\
& set(a, 1) \\
& set(b, 2) \\
& \textbf{pull}(v_a, a)\\
& \textbf{pull}(v_b, b)\\
& \textbf{assert}(v_a\neq v_b)
\end{align*}
\end{minipage}
\begin{minipage}{.3\textwidth}
\begin{align*}
set(& this, caller, val) := \\ 
& \textbf{pull}(l_v, caller, this)\\
& \textbf{push}(l_v, caller, val)\\
test(& )  := \\ 
& \textbf{push}(a, 0)\\
& \textbf{push}(b, 0)\\
& set(a, \texttt{L0}, 1) \\
& set(b, \texttt{L1}, 2) \\
& \textbf{pull}(v_a, a)\\
& \textbf{pull}(v_b, b)\\
& \textbf{assert}(v_a\neq v_b)
\end{align*}
\end{minipage}
\caption{Example of rich reference encoding that includes an ID for the current calling context. The left shows a listing where a field gets updated in a setter.
The middle shows encoding without considering the caller of \texttt{set}. The 
right column shows an encoding where a caller ID is
added to the arguments of pull and push.}
\label{fig-caller-id}
\end{figure}

Figure~\ref{fig-caller-id} shows an example of how additional arguments in
pull and push statements can help to verify programs. The left column shows
a program with a test method we want to verify that uses a setter to update a field. A verification attempt with the 
current encoding, shown in the middle column, is unlikely to be successful 
because a Horn solver will infer the summary $v=1 \vee v=2$ for the update of
\texttt{v} in the method \texttt{set}. 

To help the Horn solver find an invariant that is sufficient to verify 
the assertion, this project will explore different strategies for providing 
additional information in the pull and push statement. For this example, it is 
sufficient to add an extra argument to pull and push that keeps track of the
current calling context. The right column of Figure~\ref{fig-caller-id} shows
an example of such an encoding. Here, a unique identifier is used to distinguish
calling contexts. With this, the Horn solver will be able to identify a 
disjunctive invariant of the form 
$(caller=L0\wedge v=1) \vee (caller=L1\wedge v=2)$ that is sufficient to
verify the assertion in \texttt{test}.

A central part of the work on this project will focus on identifying data
that can be encoded into pull and push statements and on the generation of
benchmark problems that require this information to find suitable invariants.
Such benchmarks can form interesting clusters of verification problems that
will also be of interest to other tool developers.



\subsubsection{Multi-object Heap Invariants}
The first two parts of the project focus on an improved encoding of 
programs that assists a Horn solver in finding better 
heap invariants. With this encoding, we can only infer heap 
invariants that speak about one object. Invariants for properties 
such as sortedness, or contents of arrays, which reference multiple
objects, cannot be inferred using this approach.

To address this problem, this project will investigate the use of
k-indexed invariants to express properties about multiple objects.
Indexed invariants~\cite{Sanchez:2012:IGP:2414936.2414950} have been
used successfully to encode parameterized systems. The work 
in~\cite{DBLP:journals/corr/HojjatRSY14} shows how a communicating
timed system can be encoded into Horn clauses so that an 
off-the-shelf Horn solver can infer k-indexed invariants. 
A simple version of k-indexed invariants for communicating systems 
of $n$ processes $p_1,\ldots,p_n$ with a
global state $g$, and a local state $l[p_i]$ for each process 
$p_i$ can be expressed as follows: let $A_k\subseteq \{0,1\}^n$ be
the set of all bit-vectors of length $n$, where for 
any $\overline{a}\in A$ at most $k$ elements have the value $1$ and all 
other elements have value $0$. Further, let $l[\overline{a}]$ be the
sequence of local states $l[i_1], \ldots, l[i_k]$, where 
$i_1, \ldots i_k$ are the non-zero positions in $\overline{a}$.
Then, let $R_{\overline{a}}(g,l[\overline{a}])$ be a predicate that
speaks about the global state and the local states of the $k$ 
processes referenced with non-zero indices in $\overline{a}$. With
this predicate, a k-indexed invariant is then expressed as a
conjunction:
\[
Inv(g,l)= 
\bigwedge_{\overline{a}\in A_k} R_{\overline{a}}(g,l[\overline{a}])
\]
The expressibility of this type of k-invariant is limited to 
n-bounded systems with homogeneous processes. This can be extended
to unbounded systems with heterogeneous processes as discussed in~\cite{DBLP:journals/corr/HojjatRSY14}.

This project will follow this direction, but instead of encoding
multiple processes, it will encode multiple objects. The local 
states of the process then correspond to the state of the fields 
of an object. 

The current encoding of memory access into push and pull statements
can be seen as 1-indexed heap invariants. For a 2-indexed encoding,
the pull and push statements would be augmented to take a second
(rich-)reference and the values of the corresponding fields as input.

\begin{figure}[ht]
\lstset{language=Java,
numbers=left}
\centering
\begin{minipage}{.95\textwidth}
\centering
\begin{minipage}{.48\textwidth}
\begin{lstlisting}
class MyList {
  public MyList next;
  public int val;

  static void 
   verifySorted(MyList h) {
    if (h==null) return;
    while(h.next!=null) {
      MyList n = h.next
      assert(h.val<=n.val);
      h=n;
    }
  }
}
\end{lstlisting}
\end{minipage}
\begin{minipage}{.49\textwidth}
\begin{lstlisting}[firstnumber=15]
  static void createLst(int n) {
    MyList h = new MyList();
    h.val=0;
    MyList cur = h;
    for (int i=1;i<n;i++) {
      MyList nx = new MyList();
      nx.val = i;
      cur.next = nx;
      cur = nx;
    }
    verifySorted(h);
  }
}
\end{lstlisting}
\end{minipage}
\end{minipage}
\caption{Example program illustrating the concept of k-indexed invariants. The
class implements a simple list of integers with a test method \texttt{createLst}
that creates a list of length \texttt{n} and uses \texttt{verifySorted}
to ensure the result is sorted.}
\label{fig-sorted}

\lstset{language=Java,
numbers=none}
\end{figure}

Figure~\ref{fig-sorted} shows an example program where the assertion in line~10
cannot be verified without invariants that speak about multiple objects. To prove
this assertion, the Horn solver needs to be able to state that, for any two connected
list elements, the value of the predecessor is smaller or equal to the value of the 
successor.

For example, the body of \texttt{createLst} contains three references of type \texttt{MyList}: 
\texttt{h}, \texttt{cur}, and \texttt{nx}; that is, an invariant for 
\texttt{MyList} inside this procedure should be at most a 3-indexed invariant.

The memory encoding from the previous sections can be
seen as a 1-index encoding. For 1-index encoding, the pull and push statements 
will carry arguments about one object of type \texttt{MyList}, and the Horn
solver is likely to find an invariant such as $0\leq val < n$. Although this is a valid
invariant, it is not useful to prove the sortedness property.

For a 2-index encoding, the pull and push statements will receive additional
arguments for a second reference, such as
\[
\textbf{push}(type,ref_1, val_1, next_1, ref_2, val_2, next_2)
\]
For the procedure \texttt{createLst} from the example, this encoding would allow the
 Horn solver to infer invariants like:
 \[
 (ref_2=next_1 \wedge val_1\leq val_2)
 \vee (ref_1=next_2 \wedge val_2\leq val_1)
 \]

The use of k-indexed invariants raises several interesting problems, which
the proposed project plans to explore:
\begin{itemize}
\item How to identify a suitable $k$ to verify an interesting set of 
properties without sacrificing performance.
\item How to keep the inferred invariants small; e.g., by restricting search
to symmetric invariants.
\item Develop a test harness for each $k$, containing problems that cannot
be verified with a lower $k$.
\end{itemize}




