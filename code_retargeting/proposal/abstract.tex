\begin{abstract}
% The recent rise of Internet-scale code search engines---e.g., Ohloh code, Portfolio---has given search-driven development more diverse sources of information---e.g., from open source libraries to online snippets from Stackoverflow---and a more open platform---the Internet. This new condition has enabled developers to build applications opportunistically by iteratively finding, and reusing online source code. This opportunistic way of building applications is not easy. This is because search sources are large, in most cases unsuitable, and quite often unrelated. Consequently, if search-driven development were to be established as best practice, then the time involved in deciding a best search result to reuse must be minimized.
This paper sketches a research path that seeks to examine the search for suitable code problem, based on the observation that when code retargeting is included within a code search activity, developers can justify the suitability of these results upfront and thus reduce their searching efforts looking for suitable code. To support this observation, this paper introduces the Snippet Retargeting Approach, or simply \uppercase{SNIPR}. \uppercase{SNIPR} complements code search with code retargeting capabilities. These capabilities' intent is to help expedite the process of determining if a found example is a best fit. They do that by allowing developers to explore code modification ideas in place, without requiring to leave the search interface. With \uppercase{SNIPR}, developers engage in a virtuous loop where they find code, retarget code, and select only code choices they can justify as suitable. This assures immediate feedback on retargeted examples and thus saves valuable time searching for appropriate code. 

\end{abstract}



