\fancybreak{\pfbreakdisplay}

\section{Research Questions}
\label{sec:questions}

An essential part of knowing what kind of search-driven tools could help developers find the right examples to reuse is knowing precisely why search-driven development is so difficult. I will examine this issue via exploration of the following guiding questions: 

\begin{itemize}
	
	\item[RQ1] What kind of \uppercase{SnipR} operations could be invoked directly from the search 
	box? 
	
	The aim is to design a set of retargeting operations that can be invoked by developers from the 
	search box, without losing any sight of their current search goal. This design should balance 
	two inter-related principles: simplicity and flexibility. For simplicity, SNIPR will provide 
	most of the mechanisms for expression and control of changes that can be made to modules 
	(methods that implement an API). This involves discovering the core concepts of 
	retargeting modules, extracting them from their various reincarnations, and presenting them in a 
	pure and distilled form. This form will be inspired by the syntax of the Io programming 
	language\footnote{\url{http://iolanguage.org/}}. For flexibility, a simple language for combining, and 
	executing code retargeting commands will be developed. To evaluate simplicity, a user study and a survey will 
	be performed. For flexibility, the statistics on the use and execution effort for supporting and using 
	different commands will be reported.
		
	% That is, it should resemble the language used by developers when 
	% 	modifying source code, and allow a reasonable set of code changes to be expressed. By applying a 
	% 	simple but well principled rule, I would be able to decide when I have completed this question: This question is 
	% 	complete when I have stopped finding reasons to change any of the already defined commands.
	% 	
	% \item[RQ1] How does \emph{SnipR} compare to existing search-driven development 
	% 	systems~\cite{Bajracharya:2010um, Gysin:2010kt, Hartmann:2010hx, McMillan:2012dj, Sawadsky:2011eh, Wightman:2012gc} and are there any decisive factors in deciding on the approach or 
	% 	system to be used for search-driven development? 
	% 	
	% 		
	% 	I will show \emph{SnipR} is less a competitor to any of the above referenced systems and more 
	% 	of a platform. It is an effective way to improve the search for suitable 
	% 	examples, but any number of other development tools can come into play after 
	% 	the retargeting of the first few examples.  
	% 	%\emph{SnipR} will target a situation where the users don't know what they want until 
	% 	%they see it. There's no prior knowledge about the usefulness or uselessness of those 
	% 	%examples, and there is limited time and resources.
	% 	
	\item[RQ2] How expensive is it to retarget (part of) found source code each time?
	
	This question is about the performance (in terms of computational throughput) of the code 
	retargeting algorithms. In other words, this is about how long it will take the code retargeting 
	algorithms to apply the appropriate code mappings to any found source code. Code retargeting is 
	an operation that can operate on a single result or an entire result set. Therefore, this 
	operation requires that those cases where code mappings can be applied are 
	carefully identified. This will prevent any unnecessary work from happening as matched code 
	examples are being returned by the query engine. To evaluate the performance of this algorithms, 
	a user study will be performed, a set of microbenchmarks will be implemented, and the obtained statistics 
	will be reported.
	
	% Per Jim's comment on Sunday Oct 28, 2012 
	% \item[RQ2] What is the impact on programming time when retargeting is done at the 
	% integrated development environment versus when it is done at the search interface?  
	% 
	% Significant amount of time is lost when developers must move retrieved code from one 
	% environment to another. The screening and changing of the retrieved code at the 
	% search interface would minimize any back and forth movement between results 
	% evaluation and further searches.
	
	\item[RQ3] Where does \uppercase{SnipR} belong within the query processing step of any 
	modern code search engine?
	
	The answer to this question will be based on a working system for code search, such as 
	Sourcerer\cite{Bajracharya:2006vn} Internet-scale code search engine. This system will 
	be updated with \uppercase{SnipR}. Then, this work will rely on any experimentation performed 
	on the updated system to answer this question. 
	% 
	% \item[RQ3] Will productivity time increase when using SnipR versus not using SnipR?
	% 
	% Retargeting time is directly linked to integration time. If integration time goes 
	% down, then productivity time goes up. Since SnipR is trying to minimize 
	% integration time, then productivity is likely to increase.
	
	% \item[RQ4] How does the performance---in terms of the time to perform a more complete 
	% code search task---of a system using SnipR compare to systems where SnipR is not 
	% available?
	% 
	% \item[RQ4] How does the performance---in terms of the time needed to perform a more complete 
	% code search task---of the SnipR approach compare to the current approaches of code search? 
	
	\item[RQ4] How does the time needed to perform a more complete code search task of the 
	\uppercase{SnipR} approach compare to the current approaches of code search? 
	
	If the developer is retargeting the source in advance, then the evaluation results of 
	retargeted examples are available immediately (See \uppercase{SnipR} scenario in 
	Figure~\ref{fig:retargeting}). This immediate feedback saves valuable time searching for an 
	appropriate example code. As a result, by getting immediate feedback, the developer gets the 
	work done more quickly. This will hold true only if \uppercase{SnipR}'s retargeting operations 
	are efficient, which will be demonstrated by answering RQ2.
	
	% 
	%  
	% \item[RQ4] Will there be a productivity boost when using \emph{SnipR} versus not using \emph{SnipR}?
	% 
	% Retargeting in advance is directly linked to minimizing integration time. If integration time goes 
	% down, then productivity---the time required to write a fixed amount of code---goes up. Since the 
	% retarget code produced by \emph{SnipR} is the best fit, then the integration of such code is faster than the 
	% integration of less suitable code; i.e., code that relies only on ranking values. Consequently, I anticipate a 
	% boost in productivity.
		
	% \item[RQ4] Would the very way code manipulation is piggy-backed to query processing 
	% and advised by users' queries lead to a successful code integration?
	% 
	% If the most suitable results are found by retargeting, then the odds of code 
	% integration success are increased. By validating the SnipR approach 
	% and results in practice, I would be able to confirm whether the retargeted code 
	% produced by SnipR was successfully integrated.
\end{itemize}

All the work proposed in this document is motivated by a grander vision of where the next generation of programming practices and behaviors involving code search could take us. The time is ripe to change the role of code search as a development tool from a mere supporting role to a starring role.