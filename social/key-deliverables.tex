\section{Key Deliverables}
\label{sec:deliverables}

To show the feasibility of the approach we will implement a prototype of the proposed green linters that monitors at least two different \textit{IoT} open source projects. We will pick either \todo{\Daniel Please change}{xxx or yyy} as one open source project to have a baseline to compare with any existing predictors, such as \textit{Codex}, \textit{Empath}, and \textit{Augur}, etc. The final choice of the selected open source software will depend on the availability of social interaction and revision history data.

{\bfseries Preparation of training data.} To obtain traceable and reproducible results, we will use a dataset provided by GitHub Archive that contains closed to 1 TB of project-specific data, such as code, revision history, pull requests, issues, and build logs. To build our database, we will identify common programming and conversational habits automatically, then we will feed to crowdsourced expert programmers for metadata, such as intention of habit, and a descriptive text.

{\bfseries Pattern discovery from data.} \todo{\Huascar or \Daniel ?}{TBD}

{\bfseries Detection of surprisingly unlikely code and conversational habits} \todo{\Huascar or \Daniel ?}{TBD}


\todo{\Huascar Add some milestones and deliverables?}{TBD}
% We plan on the following milestones and deliverables.  \improvement{\Daniel we
% need some new milestones!}{Month 1: Data infrastructure for pre-existing code
% repositories and real-time CtF performance.  This infrastructure will allow for
% the merging of historical and CtF data, with performer information anonymized.
% Month 2: Populated data infrastructure based on historical data and quantitative
% and qualitative assessments of the CtF.  Month 6: Preliminary algorithms for
% modeling individual competence and performance, adversarial/defense
% classification, and team makeup}
